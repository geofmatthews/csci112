\documentclass{article}
\usepackage[margin=1in]{geometry}
\usepackage{fancyvrb}
\usepackage{multicol}
\usepackage{hyperref}
\usepackage{amsmath}
\usepackage{amsfonts}

\usepackage[listings]{tcolorbox}

\definecolor{codegreen}{rgb}{0,0.6,0}
\definecolor{codegray}{rgb}{0.5,0.5,0.5}
\definecolor{codepurple}{rgb}{0.58,0,0.82}
\definecolor{backcolour}{rgb}{0.95,0.95,0.92}

\lstdefinestyle{mystyle}{
    language=Python,
    backgroundcolor=\color{backcolour},   
    commentstyle=\color{codegreen},
    keywordstyle=\color{magenta},
    numberstyle=\tiny\color{codegray},
    stringstyle=\color{codepurple},
    basicstyle=\ttfamily\footnotesize,
    breakatwhitespace=false,         
    breaklines=true,                 
    captionpos=b,                    
    keepspaces=true,                 
    numbers=left,                    
    numbersep=5pt,                  
    showspaces=false,                
    showstringspaces=false,
    showtabs=false,                  
    tabsize=2,
    escapechar=|,
    frame=single
}

\lstset{style=mystyle}

\newcommand{\showfig}[2]{
\noindent\includegraphics[width=\textwidth]{#1}
\centerline{#1}
}
\newcommand{\bi}{\begin{itemize}}
\newcommand{\li}{\item}
\newcommand{\ei}{\end{itemize}}

\title{Eliminating Recursion with Stacks}
\author{CSCI 112, Lab 6}
\date{}

\begin{document}
\sloppy

\maketitle

\begin{description} 
\item[File names:]  Names of files, functions, and variables, 
when specified,
must be EXACTLY as specified.  This includes simple mistakes such
as capitalization.

\item[Individual work:]  All work must be your own.  Do not share
code with anyone other than the instructor and teaching assistants.
This includes looking over shoulders at screens with the code open.
You may discuss ideas, algorithms, approaches, {\em etc.} with
other students but NEVER actual code.  Do not use code
written by anyone else, in the class or from the internet.

\item[Documentation:] Each file should begin with a docstring
that includes your name, the class number and name, the lab
number, and  
a short description of the lab, as well as documentation pertinent
to that particular file.

\item[Translations:]  Translate each of the following
recursive functions into ``unnested'' versions and into
non-recursive functions following
the style of the functions discussed in class and available in the
file \lstinline{stackrecursion.py}.  Put all the functions into
a file called \lstinline{uncursion.py} and unit tests
for all of them into a file called \lstinline{uncursion_test.py}.

Put everything into a folder names {\tt csci112lab06yourname}, zip
and turn into canvas.

\item[The problems:]
\end{description}

\begin{enumerate}

\li
\begin{lstlisting}
def func01(n):
    if n < 2:
        return 3*n
    else:
        return 2*func01(n-1)
\end{lstlisting}


\li
\begin{lstlisting}
def func02(n):
    if n < 2:
        return 3*n
    elif n < 6:
        return n*n
    else:
        return 3 + 2*func02(n-2)
\end{lstlisting}


\li
\begin{lstlisting}
def func03(n):
    if n < 2:
        return 3*n
    else:
        return 3*func03(n-1) + 2*func03(n-2)
\end{lstlisting}


\li
\begin{lstlisting}
def func04(n):
    if n < 2:
        return 3*n
    elif n%2 == 1:
        return 7 + func04(n - 3)
    else:
        return func04(n-1) * 2*func04(n//2)
\end{lstlisting}


\li
\begin{lstlisting}
def func05(a, b):
    if a == 0:
        return b+3
    elif b == 0:
        return a*2
    else:
        return 2*a + 3*b + 4*func05(a-1,b) + 5*func05(a, b-1)
\end{lstlisting}


\end{enumerate}


\end{document}
