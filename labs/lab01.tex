\documentclass{article}
\usepackage[margin=1in]{geometry}
\usepackage{fancyvrb}
\usepackage{multicol}
\usepackage{hyperref}
\usepackage{amsmath}
\usepackage{amsfonts}

\usepackage[listings]{tcolorbox}

\definecolor{codegreen}{rgb}{0,0.6,0}
\definecolor{codegray}{rgb}{0.5,0.5,0.5}
\definecolor{codepurple}{rgb}{0.58,0,0.82}
\definecolor{backcolour}{rgb}{0.95,0.95,0.92}

\lstdefinestyle{mystyle}{
    language=Python,
    backgroundcolor=\color{backcolour},   
    commentstyle=\color{codegreen},
    keywordstyle=\color{magenta},
    numberstyle=\tiny\color{codegray},
    stringstyle=\color{codepurple},
    basicstyle=\ttfamily\footnotesize,
    breakatwhitespace=false,         
    breaklines=true,                 
    captionpos=b,                    
    keepspaces=true,                 
    numbers=left,                    
    numbersep=5pt,                  
    showspaces=false,                
    showstringspaces=false,
    showtabs=false,                  
    tabsize=2,
    escapechar=|,
    frame=single
}

\lstset{style=mystyle}

\newcommand{\showfig}[2]{
\noindent\includegraphics[width=\textwidth]{#1}
\centerline{#1}
}
\newcommand{\bi}{\begin{itemize}}
\newcommand{\li}{\item}
\newcommand{\ei}{\end{itemize}}

\title{FibLog}
\author{CSCI 112, Lab 1}
\date{}

\begin{document}
\sloppy

\maketitle

\begin{description} 
\item[File names:]  Names of files, functions, and variables, 
when specified,
must be EXACTLY as specified.  This includes simple mistakes such
as capitalization.

\item[Individual work:]  All work must be your own.  Do not share
code with anyone other than the instructor and teaching assistants.
This includes looking over shoulders at screens with the code open.
You may discuss ideas, algorithms, approaches, {\em etc.} with
other students but NEVER actual code.  Do not use code
written by anyone else, in the class or from the internet.

\item[Documentation:] Each file should begin with a docstring
that includes your name, the class number and name, the lab
number, and  
a short description of the lab, as well as documentation pertinent
to that particular file.

\item[Matrix:]  Write a \lstinline{Matrix} class, in a module \lstinline{matrix.py}.
The initialization should take the number of rows and columns,
and then an iterable of length \lstinline{rows * cols}.
  It should 
implement at least the following methods:
\bi
\li \lstinline{__init__}
\li \lstinline{__str__}
\li \lstinline{__mul__}
\li \lstinline{__pow__}
\ei
Which should enable the following interactions:
\begin{lstlisting}
>>> from matrix import Matrix
>>> m1 = Matrix(3, 4, range(3*4))
>>> print(m1)
|    0    1    2    3|
|    4    5    6    7|
|    8    9   10   11|
>>> print(m1 * Matrix(4,3,range(4*3)))
|   42   48   54|
|  114  136  158|
|  186  224  262|
>>> m2 = Matrix(2, 2, [0,1,1,1])
>>> print(m2)
|    0    1|
|    1    1|
>>> print(m2 * m2)
|    1    1|
|    1    2|
>>> print(m2 ** 2)
|    1    1|
|    1    2|
>>> m3 = Matrix(2, 2, 'abcd')
>>> print(m3)
|    a    b|
|    c    d|
\end{lstlisting}

\item[Unit test:]  Implement a unit test module for your matrix class. Call
the unit test module \lstinline{matrix_test.py}.

\item[Fibonacci:]  In a module named \lstinline{fibonacci.py},
implement the Fibonacci function three different ways.  One will
execute in exponential time, $O(2^n)$, one in linear time, $O(n)$, and one in log time
$O(\log n)$.
Mathematically, the fibonacci function is defined as follows.
defined as 
\[
f(n) = \left\{\begin{array}{ll}
     0 & \mbox{if } n = 0\\
       1 & \mbox{if } n = 1\\
       f(n-1) + f(n-2) & \mbox{otherwise}\end{array}\right.
 \]
 The log time algorithm should be based on calculating
 the powers of the matrix
 \[
 \left[\begin{array}{cc}0 & 1 \\ 1 & 1 \end{array}\right]
 \]
 
 \item[Unit test:] Implement a unit test module for your
 Fibonacci functions.  Call the unit test module \lstinline{fibonacci_test.py}.
 
 \item[Timing:] Finally, implement a module called \lstinline{fibonacci_timing.py}
 to test the runtime
 of your three fibonacci functions.  Use the \lstinline{time.process_time_ns}
 function to get more accurate timings, in nanoseconds.  Design
 tests to show the different times of the different functions and illustrate
 the differences between $O(2^n)$, $O(n)$, and $O(\log n)$.
 
 \item[Writeup:]  Write a short paper discussing your findings from the
 timing experiments, and why they support (or don't!) our order of
 magnitude estimates of their runtimes. 
 
 This writeup must be in either a plain text or a \LaTeX\ document.
 \LaTeX is optional, but I highly recommend that you learn to use
 it as soon as possible.  There are many online tutorials, and {\em all}
 of my lab notes and lectures are written in \LaTeX!  A good place to
 get started in \LaTeX\ is \url{https://www.overleaf.com/}.
 
 Put this writeup in a file
 called either {\tt lab01writeup.txt}
 or  {\tt lab01writeup.tex}
 
 If you use \LaTeX\ you may leave in place the miscellaneous
 files produced on compilation to pdf format. 
 
 \item[Turn in:] Put the following files into a folder called {\tt lab01csci112},
 zip it into a single compressed file and submit to canvas:

\bi
\li \lstinline{matrix.py}
\li \lstinline{matrix_test.py}
\li \lstinline{fibonacci.py}
\li \lstinline{fibonacci_test.py}
\li \lstinline{fibonacci_timing.py}
\li Either:
\bi
\li \lstinline{lab01writeup.txt}
\li \lstinline{lab01writeup.tex}
\ei
\ei
 


\end{description}



\end{document}
