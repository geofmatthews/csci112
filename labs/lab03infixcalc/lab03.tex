\documentclass{article}
\usepackage[margin=1in]{geometry}
\usepackage{fancyvrb}
\usepackage{multicol}
\usepackage{hyperref}
\usepackage{amsmath}
\usepackage{amsfonts}

\usepackage[listings]{tcolorbox}

\definecolor{codegreen}{rgb}{0,0.6,0}
\definecolor{codegray}{rgb}{0.5,0.5,0.5}
\definecolor{codepurple}{rgb}{0.58,0,0.82}
\definecolor{backcolour}{rgb}{0.95,0.95,0.92}

\lstdefinestyle{mystyle}{
    language=Python,
    backgroundcolor=\color{backcolour},   
    commentstyle=\color{codegreen},
    keywordstyle=\color{magenta},
    numberstyle=\tiny\color{codegray},
    stringstyle=\color{codepurple},
    basicstyle=\ttfamily\footnotesize,
    breakatwhitespace=false,         
    breaklines=true,                 
    captionpos=b,                    
    keepspaces=true,                 
    numbers=left,                    
    numbersep=5pt,                  
    showspaces=false,                
    showstringspaces=false,
    showtabs=false,                  
    tabsize=2,
    escapechar=|,
    frame=single
}

\lstset{style=mystyle}

\newcommand{\showfig}[2]{
\noindent\includegraphics[width=\textwidth]{#1}
\centerline{#1}
}
\newcommand{\bi}{\begin{itemize}}
\newcommand{\li}{\item}
\newcommand{\ei}{\end{itemize}}

\title{Infix Calculator}
\author{CSCI 112, Lab 3}
\date{}

\begin{document}
\sloppy

\maketitle

\begin{description} 
\item[File names:]  Names of files, functions, and variables, 
when specified,
must be EXACTLY as specified.  This includes simple mistakes such
as capitalization.

\item[Individual work:]  All work must be your own.  Do not share
code with anyone other than the instructor and teaching assistants.
This includes looking over shoulders at screens with the code open.
You may discuss ideas, algorithms, approaches, {\em etc.} with
other students but NEVER actual code.  Do not use code
written by anyone else, in the class or from the internet.

\item[Documentation:] Each file should begin with a docstring
that includes your name, the class number and name, the lab
number, and  
a short description of the lab, as well as documentation pertinent
to that particular file.

\item[Calculator:] Finish the following two problems from the text:
\begin{enumerate}
\item
Implement a direct infix evaluator that combines the functionality of infix-to-postfix conversion and the postfix evaluation algorithm. Your evaluator should process infix tokens from left to right and use two stacks, one for operators and one for operands, to perform the evaluation.

Follow the textbook's style.  Try to use as much of the code from the textbook
as possible.  You should be able to simply modify the \lstinline{infixToPostfix}
function using the \lstinline{doMath} function.  Every time the function would
output an operator, apply this operator to the top two items on the stack.
Think about the order of operands and the order of stacking!
\item
Turn your direct infix evaluator from the previous problem into a calculator.
This will simply print a prompt, read the input, evaluate the arithmetic 
expression, print the result, and loop.  Here's mine in action:
\begin{Verbatim}[frame=single]
>>> repl()
Geoff's Amazing Calculator! >>> 2 + 2
4
Geoff's Amazing Calculator! >>> 3 + 4 * 5
23
Geoff's Amazing Calculator! >>> 3 * 4 + 5
17
Geoff's Amazing Calculator! >>> quit
>>> 
\end{Verbatim}
{\tt repl} stands for ``read, evaluate, print, loop''

\end{enumerate}


\item[Tokenizing:]  Instead of using single-character numbers, as in the text,
use my tokenizing function found in \lstinline{tokens.py}.  This allows the use
of floats, too.   It uses \lstinline{split()} to separate tokens, so all
tokens must be separated by spaces, including operators and parentheses.
For example:
\begin{lstlisting}
>>> tokenize('( 44 + -33.3 ) * 2 - 4')
['(', 44, '+', -33.3, ')', '*', 2, '-', 4]
\end{lstlisting}


\item[Operators:] You must support addition, subtraction, multiplication
and division.  Multiplication and division have higher precedence than 
addition and subtraction.  Otherwise everything is left associative.

\item[Turn in:]  A file named \lstinline{calculator.py}, with a function
called \lstinline{evaluate} with the following behavior, which does infix
evaluation, for example:
\begin{lstlisting}
evaluate(`2 + 2')   =>   4
\end{lstlisting}
and a function called \lstinline{repl} which starts a read-eval-print loop.

A unit test module called \lstinline{calc_test.py}, that tests the major
functionality of your calculator functions.

Zip these functions, together with \lstinline{tokens.py} and any other
modules you built, in a folder called \lstinline{csci112lab03yourname} zip and submit to canvas.

\item[Optional additions:]~ 

\bi
\li Write a better tokenizer so that spaces are not required where they are not
needed, for example, 
\fbox{\lstinline{(2+2)}}
\li Add error checking to the tokenizer, so that the user gets meaningful
feedback from expressions like these:
\fbox{\lstinline{3 + )}}, 
\fbox{\lstinline{4 + * 5}}, and
\fbox{\lstinline{( 3 ( 4 + 5 ))}},
and the calculator continues to work, rather than stopping with an error.


\li Add some unary operators, such as \lstinline{sin} and \lstinline{cos} and \lstinline{exp}


\li Add the \lstinline{**} operator.  Note that this has higher precedence
than all the others, and is also right associative.  You will need to think through
how to handle the operator stack for this one, and make the necessary
additions to the algorithm outlined in \S 4.9.2.  It is not easy.

Include  a separate document explaining how your
algorithm correctly handles right associative operators.




\ei


\end{description}



\end{document}
