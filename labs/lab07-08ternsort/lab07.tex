\documentclass{article}
\usepackage[margin=1in]{geometry}
\usepackage{fancyvrb}
\usepackage{multicol}
\usepackage{hyperref}
\usepackage{amsmath}
\usepackage{amsfonts}

\usepackage[listings]{tcolorbox}

\definecolor{codegreen}{rgb}{0,0.6,0}
\definecolor{codegray}{rgb}{0.5,0.5,0.5}
\definecolor{codepurple}{rgb}{0.58,0,0.82}
\definecolor{backcolour}{rgb}{0.95,0.95,0.92}

\lstdefinestyle{mystyle}{
    language=Python,
    backgroundcolor=\color{backcolour},   
    commentstyle=\color{codegreen},
    keywordstyle=\color{magenta},
    numberstyle=\tiny\color{codegray},
    stringstyle=\color{codepurple},
    basicstyle=\ttfamily\footnotesize,
    breakatwhitespace=false,         
    breaklines=true,                 
    captionpos=b,                    
    keepspaces=true,                 
    numbers=left,                    
    numbersep=5pt,                  
    showspaces=false,                
    showstringspaces=false,
    showtabs=false,                  
    tabsize=2,
    escapechar=|,
    frame=single
}

\lstset{style=mystyle}

\newcommand{\showfig}[2]{
\noindent\includegraphics[width=\textwidth]{#1}
\centerline{#1}
}
\newcommand{\bi}{\begin{itemize}}
\newcommand{\li}{\item}
\newcommand{\ei}{\end{itemize}}

\title{Ternary Mergesort and Quicksort}
\author{CSCI 112, Labs 7 and 8}
\date{}

\begin{document}
\sloppy

\maketitle

\begin{description} 
\item[File names:]  Names of files, functions, and variables, 
when specified,
must be EXACTLY as specified.  This includes simple mistakes such
as capitalization.

\item[Individual work:]  All work must be your own.  Do not share
code with anyone other than the instructor and teaching assistants.
This includes looking over shoulders at screens with the code open.
You may discuss ideas, algorithms, approaches, {\em etc.} with
other students but NEVER actual code.  Do not use code
written by anyone else, in the class or from the internet.

\item[Documentation:] Each file should begin with a docstring
that includes your name, the class number and name, the lab
number, and  
a short description of the lab, as well as documentation pertinent
to that particular file.

\item[The project:]  Each of the sorting algorithms mergesort and quicksort
divides the list into two portions, sorts the portions, and then puts them
back together. 

Rewrite each of these algorithms so that they divide the list into three
portions, sort the portions, and then put them back together.

Use the implementation that comes with the textbook as a start.
Try to make as few changes as possible.  Mergesort should be
fairly easy.  

Quicksort's partition function will be more difficult.
You will have to select two pivots, $p1$ and $p2$, and then
arrange the list into three sections:
\begin{align*}
n &< p1\\
p1 \leq n &< p2\\
p2 &<= n
\end{align*}
There is a fairly easy way to do it if you let yourself traverse
the array twice, instead of once, to make the partition.
(Think about using the original partition algorithme, twice.)
This, of course, will make the partition algorithm $O(2n)$,
but of course, $2n = O(n)$.
Implement this first so that you can create a unit test module
and test your sorting implementations.

\item[Lab 6:]  That's it for lab 6.   Put all four sorting algorithms (mergesort and quicksort, binary and
tertiary) in a single file called \verb|sorts.py|.  Put your unit tests in the file \verb|sorts_test.py|.  Zip these files in a folder \verb|csci112lab06yourname|
and turn in to canvas.

\item[Lab 7:]  For lab 7 develop a better quicksort ternary partition 
algorithm as follows, and run some timing tests on all four algorithms.

\item[Improved quicksort partition:] Develop a quicksort ternary partition that traverses
the array only once, and partitions in place ({\em i.e.} using constant extra storage,
that doesn't grow with the array size).  To partition
an array between indices \lstinline{low} and \lstinline{hi}, you will
need two pivots, which will be elements at \lstinline{low} and \lstinline{hi}.  If the
first element is larger than the last, swap them first.

Then, the idea is to create three ranges of elements:
\[ range1 < pivot1 \leq range2 < pivot2 \leq range3 \]
You can do this with three indices into the array: \lstinline{left}, \lstinline{right},
which start at \lstinline{low+1} and \lstinline{hi-1}, and \lstinline{i} which starts
out at the same place as \lstinline{left}, and moves gradually to the right.

When \lstinline{i >= right} the list will be partitioned.  At each step,
you need to compare \lstinline{a[i]} with both \lstinline{a[left]}
and \lstinline{a[right]}.  Depending on the three possible outcomes
of these comparisons, you need to move one or more of
\lstinline{right}, \lstinline{left}, and \lstinline{i}.

Think carefully about how this has to be done to accomplish the
partitioning required.  Implement several variations and test them
until you're sure you've got it right.

A trace of the algorithm as I implemented it is shown below:

\begin{tabular}{|c|c|c|c|c|c|c|c|c|c|c|l|}\hline
     low&  left,i&        &        &        &        &        &        &        &        &   right&      hi\\\hline
       3&       9&       4&       5&       1&       5&       1&       9&      10&       8&       2&       9\\\hline\hline
     low&  left,i&        &        &        &        &        &        &        &   right&        &      hi\\\hline
       3&       2&       4&       5&       1&       5&       1&       9&      10&       8&       9&       9\\\hline\hline
     low&        &  left,i&        &        &        &        &        &        &   right&        &      hi\\\hline
       3&       2&       4&       5&       1&       5&       1&       9&      10&       8&       9&       9\\\hline\hline
     low&        &    left&       i&        &        &        &        &        &   right&        &      hi\\\hline
       3&       2&       4&       5&       1&       5&       1&       9&      10&       8&       9&       9\\\hline\hline
     low&        &    left&        &       i&        &        &        &        &   right&        &      hi\\\hline
       3&       2&       4&       5&       1&       5&       1&       9&      10&       8&       9&       9\\\hline\hline
     low&        &        &    left&        &       i&        &        &        &   right&        &      hi\\\hline
       3&       2&       1&       5&       4&       5&       1&       9&      10&       8&       9&       9\\\hline\hline
     low&        &        &    left&        &        &       i&        &        &   right&        &      hi\\\hline
       3&       2&       1&       5&       4&       5&       1&       9&      10&       8&       9&       9\\\hline\hline
     low&        &        &        &    left&        &        &       i&        &   right&        &      hi\\\hline
       3&       2&       1&       1&       4&       5&       5&       9&      10&       8&       9&       9\\\hline\hline
     low&        &        &        &    left&        &        &       i&   right&        &        &      hi\\\hline
       3&       2&       1&       1&       4&       5&       5&       8&      10&       9&       9&       9\\\hline\hline
     low&        &        &        &    left&        &        &        & i,right&        &        &      hi\\\hline
       3&       2&       1&       1&       4&       5&       5&       8&      10&       9&       9&       9\\\hline\hline
     low&        &        &        &    left&        &        &   right&       i&        &        &      hi\\\hline
       3&       2&       1&       1&       4&       5&       5&       8&      10&       9&       9&       9\\\hline\hline
     low&        &        &    left&        &        &        &        & i,right&        &        &      hi\\\hline
       1&       2&       1&       3&       4&       5&       5&       8&       9&       9&       9&      10\\\hline
\end{tabular}

\item[Testing runtime:]  The runtime of binary and tertiary quicksort and mergesort should
be very close to each other, all of them being $O(\log n)$.  The ternary algorithms
should be slightly faster since $\log_3(n) < \log_2(n)$, but since they have more
overhead they may be slightly slower, at least for small lists.
It is possible that they will show some improvement over the binary versions
as the array gets largs.  Run some tests to see if you can get consistent results
as to whether tertiary is faster or slower than binary, or about the same.  Write up
your findings in a short document called either \lstinline{sorts_runtime.txt} or
\lstinline{sorts_runtime.tex}.  Please do not use any other word processing:
either plain ascii text or \LaTeX.

\item[Fair tests:] Make sure you instrument your tests so that each algorithm
is used on the same starting array.  If you use random arrays then some algorithms
may ``get lucky'' and have easier starting arrays than others, skewing the results.

\item[File names:]  Put all four sorting algorithms (mergesort and quicksort, binary and
tertiary) in a single file called \verb|sorts.py|.  Put your unit tests in the file \verb|sorts_test.py|.
Put your runtime tests in a file called \verb|sorts_runtime.py|.  Zip all three together with 
your writeup into a folder
called \verb|csci112lab07yourname| and submit to canvas.

\end{description}


\end{document}
