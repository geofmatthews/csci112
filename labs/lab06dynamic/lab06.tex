\documentclass{article}
\usepackage[margin=1in]{geometry}
\usepackage{fancyvrb}
\usepackage{multicol}
\usepackage{hyperref}
\usepackage{amsmath}
\usepackage{amsfonts}

\usepackage[listings]{tcolorbox}

\definecolor{codegreen}{rgb}{0,0.6,0}
\definecolor{codegray}{rgb}{0.5,0.5,0.5}
\definecolor{codepurple}{rgb}{0.58,0,0.82}
\definecolor{backcolour}{rgb}{0.95,0.95,0.92}

\lstdefinestyle{mystyle}{
    language=Python,
    backgroundcolor=\color{backcolour},   
    commentstyle=\color{codegreen},
    keywordstyle=\color{magenta},
    numberstyle=\tiny\color{codegray},
    stringstyle=\color{codepurple},
    basicstyle=\ttfamily\footnotesize,
    breakatwhitespace=false,         
    breaklines=true,                 
    captionpos=b,                    
    keepspaces=true,                 
    numbers=left,                    
    numbersep=5pt,                  
    showspaces=false,                
    showstringspaces=false,
    showtabs=false,                  
    tabsize=2,
    escapechar=|,
    frame=single
}

\lstset{style=mystyle}

\newcommand{\showfig}[2]{
\noindent\includegraphics[width=\textwidth]{#1}
\centerline{#1}
}
\newcommand{\bi}{\begin{itemize}}
\newcommand{\li}{\item}
\newcommand{\ei}{\end{itemize}}

\title{Dynamic Programming}
\author{CSCI 112, Lab 6}
\date{}

\begin{document}
\sloppy

\maketitle

\begin{description} 
\item[File names:]  Names of files, functions, and variables, 
when specified,
must be EXACTLY as specified.  This includes simple mistakes such
as capitalization.

\item[Individual work:]  All work must be your own.  Do not share
code with anyone other than the instructor and teaching assistants.
This includes looking over shoulders at screens with the code open.
You may discuss ideas, algorithms, approaches, {\em etc.} with
other students but NEVER actual code.  Do not use code
written by anyone else, in the class or from the internet.

\item[Documentation:] Each file should begin with a docstring
that includes your name, the class number and name, the lab
number, and  
a short description of the lab, as well as documentation pertinent
to that particular file.

\item[Substring:] One string is a {\em substring} of another
if the characters in the first string occur, in order, somewhere in
the second string.  For example, the string ``geoff'' is a substring
of the string ``xxgxxeoxxfxxxf''.  We will write a boolean
function that returns {\tt True} or {\tt False} depending on
whether the first argument is a substring of the second.

\item[Recursive solution:] Code up a recursive solution to
this problem using the following strategy:
\begin{itemize}
\item If the first string is empty then return {\tt True}
\item If the second string is empty then return {\tt False}
\item If the first letter of the first string does {\bf not}
match the first letter of the second string, then
recursively try to match the first string with the
second string without its first letter.
\item If the first letter of the first string {\bf does}
match the first letter of the second string,
then check recursion both with and without
matching the first letters.  If either one succeeds,
then the case succeeds.
\end{itemize}

\item[Dynamic programming:] Code up a non-recursive
solution to this problem by building up an $m\times n$ table of results
matching each letter of the potential substring with the superstring.
For example, trying to determine whether ``abc'' is  a substring
of ``xabbxc'' we get the following table, where 1 is {\tt True} 
and 0 is {\tt False}:

\begin{tabular}{cccccccc}
  &    & c & a & b& b & x & c \\\hline
  & 1 & 1 & 1 & 1 & 1 & 1 & 1 \\
a & 0 & 0 & 1 & 1 & 1 & 1 & 1 \\
b & 0 & 0 & 0 & 1 & 1 & 1 & 1 \\
c & 0 & 0 & 0 & 0 & 0 & 0 & 1 \\
\end{tabular}

Each cell in the table represents whether the string at the left
(up to the current row) is a substring of the string at the top
(up to the current column).  This can be determined from
the two current characters in the strings, and the entries
in the table above and to the left of the current entry, so
the table can be filled in with a dynamic programming 
algorithm.

\item[Turn in:]  Two procedures: \verb|is_substring_recursive|
and \verb|is_substring_dynamic|, defined in a file called \verb|substring.py|.
Also a unittest module called \verb|substring_test.py|
with randomly generated test strings.  Put all in a folder
called \verb|csci112lab06yourname|, zip and turn into canvas.

\end{description}


\end{document}
