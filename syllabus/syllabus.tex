\documentclass[12pt]{article}
\usepackage[margin=1in]{geometry}
\usepackage{hyperref}
\usepackage{alltt}
\usepackage{color}
\newcommand{\bi}{\begin{itemize}}
\newcommand{\li}{\item}
\newcommand{\ei}{\end{itemize}}
\begin{document}
\centerline{\Large Computer Science 112}
\centerline{\large Fundamentals of Programming 2}

\begin{description}
\item[Instructor:]
~\\ Geoffrey Matthews, Parmly 407A, 540-458-8809
\\ gmatthews@wlu.edu


\item[Webpages:]~
\begin{itemize}
\item For submitting labs, exams, and grades: \url{canvas.wlu.edu}
\item For lecture notes and lab descriptions: \\
\url{https://github.com/geofmatthews/csci112}

If you don't know how to use git, just go to that website,
click the \fbox{\sf Code} button, and then \fbox{\sf Download ZIP}.
This does just what it says.
\item For software: \url{https://www.python.org/}
This software is installed in the labs, but you may want to
download it for your own computer.  It's free.

\end{itemize}
\item[Class lectures:]  MWF 2:45-3:45, Parmly 405

Lectures are in-person and attendance is required.  If you must miss a
lecture, inform your instructor before the day you must miss
and arrange to get notes from another classmate.  Office hours
will be used to answer questions, but material already presented
in class will not be reviewed.

\item[Labs:]  Thursday 1:30-4:30,  Parmly 405

Labs are in-person and attendance is required.
No new material will be presented in labs, but it is a 
unique opportunity to work on the homework with direct 
assistance from the instructor and the TAs.  If you miss
a lab, you will have to make the time up yourself, and it
will likely take you longer.

\item[Office hours:] MWF 1:15-2:15 , Parmly 407A

If you need to see me but
cannot make these hours, please  make an appointment.

\item[Overview:]
This is the second introductory course in programming and problem
solving.  Topics include:
\bi
\li An introduction to {\bf algorithm analysis}, the mathematical study
of fundamental properties of computer programs.
\li {\bf Linear data structures}, including lists, stacks, queues and deques
and their applications.
\li {\bf Sorting algorithms}, including $O(n^2)$ sorts such as bubble sort,
selection sort, and insertion sort, $O(n\log n)$ sorts such as merge sort
and quick sort, and possibly some in between, such as shell sort.
\li {\bf Hashing} including both open and closed implementations
of collision resolution. 
\li {\bf Tree data structures}, basic algorithms and a sampling of their applications. 
Topics may include: parse trees, priority queues, heaps, binary search trees,
and balanced binary search trees.
\li {\bf Graph data structures} basic algorithms such as topological sorting,
strongly connected components, search algorithms such as Dijkstra's and $A^*$,
and spanning tree algorithms.
\ei

\item[Textbook:]
\href{https://runestone.academy/ns/books/published/pythonds/index.html}
{\em Problem Solving with Algorithms and Data Structures
Using Python}, by Miller and Rahum.

This book is available for free online.  
It is also available in the bookstore and at
amazon.com for those who would prefer a paper copy.
The online and paper editions have slight differences, but
either one should be fine.

There are many other online resources for studying
data structures, feel free to use as many as you think helpful!


\item[Labs:] Labs are on Thursdays.  Each lab will be due
the following Tuesday before midnight.  There will be no
late work accepted.   
There will be 11 labs, each worth a maximum of 5 points.  Your worst lab score will be 
thrown out.
Points will be awarded according to the
following.


\begin{tabular}{p{0.75\textwidth}|c}
\bf Factors to consider: & \bf Points \\\hline
{\bf Outstanding work.}    Well formatted, modular,
well commented, well designed.  
Clear, self-documenting identifiers: variables, functions, class names.
Good, consistent docstrings.
 Extra work on optional problems or extensions
of the required work.  Error checking.  Comprehensive unit tests. Innovative solutions.
Extensive documentation on design decisions and results.
 & 5 \\\hline
{\bf Good work.}  The problem is solved completely and without errors.
Adequate documentation.
 & 4 \\\hline
{\bf Adequate work.}  Most, but not all of the problem is solved.  Poor documentation.
 & 3 \\\hline
 {\bf Incomplete work.}  Some progress was made, but no complete solution.
 Nonexistent documentation.
 & 2 \\\hline
{\bf Poor work.}  Little or none of the problem is solved.  Random bits of code copied
from lectures or the problem description without showing any real coherence
or understanding of an approach to the problem.
 & 1 \\\hline
{\bf Unacceptable work.}
 Syntax errors.  Not turned in on time.  Did not follow instructions. & 0 \\
\end{tabular}

\item[Assessment survey:]
The first lab also includes filling out the Assurance of Learning Survey for CSCI Majors.
Completing this survey is {\bf required} to pass this course.

Your answers on this survey will not affect your grade in any of your courses.
By filling out this survey now and after you complete the requirements for the CSCI
major or minor, you will help us assess how well we are reaching our departmental
learning goals.

At this point we do not expect you to know the answers to most of these questions;
we, therefore, encourage you to skip questions about material that you haven’t yet
studied.

The survey can be found on canvas under assignments.

\item[Midterms:]  There will be two midterms, each worth 10\%, as
in the class schedule below.   They is open
book and open notes, but you may not consult with any classmates
or the internet or other resources.


\item[Final exam:]  The final exam is comprehensive.  It is open
book and open notes, but you may not consult with any classmates
or the internet or other resources.


\item[Grades:]  Possible points:

\begin{center}
\begin{tabular}{c|ccc}
 & \multicolumn{3}{c}{Exams}\\
Labs & Midterm 1 & Midterm 2 & Final \\
50 & 10 & 10 & 30 \\
\end{tabular}
\end{center}

Grades will be based on the following percentages
out of all possible points:
\[
A \geq 90\% > B \geq 80\% > C \geq 70\% > D \geq 60\% > F
\]
The instructor reserves the right to adjust the scale,
but only in a manner that would reward higher grades than
those predicted from the table. Awarding $\pm$ is also at
the discretion of the instructor.


\item[Computer use in class:]
The use of laptops and mobile computing devices are 
permitted during class so long as they are being used 
for the course such as for taking notes and locating
information related to the course. These devices are
not to be used during class for texting, phone calls, 
reading email, social networking, completing assignments
for other courses, shopping, or any other topic unrelated
to the class you are currently attending.

\item[Accommodations]
Washington and Lee University makes reasonable academic 
accommodations for qualified students with disabilities. 
All undergraduate accommodations must be approved through 
the Office of the Dean of the College. Students requesting 
accommodations for this course should present an official 
accommodation letter within the first two weeks of the 
(fall or winter) term and schedule a meeting outside of 
class time to discuss accommodations. It is the student’s 
responsibility to present this paperwork in a timely 
fashion and to follow up about accommodation arrangements. 
Accommodations for test-taking should be arranged with the 
professor at least a week before the date of the test or exam.

     
\item[Academic dishonesty:] Please review the university's
honor system, and the definition of plagiarism
which can be found at

\centerline{
\url{https://my.wlu.edu/executive-committee/the-honor-system}
}

  Unless specified otherwise, all work for this course is meant to
  be done {\bf individually.}  The work that you turn in for a grade
  must be completely your own, or you will be guilty of academic
  dishonesty.

  Nevertheless, it is a valiable learning experience to discuss
  work with your fellow students, and this is encouraged.
  However, after working with a colleague, {\bf you may not keep any
    paper or electronic copies of anything you produced together!}
  You may only keep your memories.  In particular, this means that
  {\bf you may not ask for or give help while sitting in front of a
    computer where the assignment is open!}  Also, {\bf you may not
    use anything a colleague has emailed to you!}  Delete the email
  and do not save a copy.

  To help understand what I mean, remember the \\
  \centerline{\fbox{\sf Long Term
    Memory Rule}}  You may discuss, sketch, write things down, use
  your computers, whatever, but after you are done working with your
  fellow students all files must be deleted, whiteboards erased, and
  all papers you created must be destroyed.  You should then watch a
  rerun of {\em the Simpson's}, play a game of ping-pong, take a walk,
  or something else for half an hour. After this you can go back to
  your assignment (alone) and use the knowledge you have now gained.

  We are here to help you get a great education.  Please do not
  put us in a situation where we have to police you for plagiarism.
  We hate that.

\newpage
\item[Schedule:]~

\begin{alltt} 
    January 2023
Su Mo Tu We Th Fr Sa
 8  9 10 11 12 13 14     Python review, Algorithm analysis
15 {\color{red}16} 17 18 19 20 21     Linear data structures
22 23 24 25 26 27 28     Recursion
29 30 31                 Review & Midterm 1
    February 2023
Su Mo Tu We Th Fr Sa
          1  2  3  4     Review & Exam
 5  6  7  8  9 10 11     Searching, Sorting
12 13 14 15 16 17 18     Trees
19 {\color{red}20 21 22 23 24} 25     Holiday
26 27 28                 Trees
     March 2023
Su Mo Tu We Th Fr Sa
          1  2  3  4     Trees
 5  6  7  8  9 10 11     Review & Midterm 2
12 13 14 15 16 17 18     Graphs
19 20 21 22 23 24 25     Graphs
26 27 28 29 30 31        Misc. topics
     April 2023
Su Mo Tu We Th Fr Sa
                   1
 2  3  4  5  6  7  8     Review
 9 {\color{red}10 11 12 13 14} 15     Final exam
\end{alltt}


\end{description}

\end{document}