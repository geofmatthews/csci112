
\documentclass[slidestop,xcolor=pst]{beamer}
\usepackage{fancyvrb}


\newcommand{\sect}[1]{
\section{#1}
\begin{frame}[fragile]\frametitle{#1}
}

\mode<presentation>
{
  \usetheme{Madrid}
  % or ...

  \setbeamercovered{transparent}
  % or whatever (possibly just delete it)
}

\usepackage[english]{babel}

\usepackage[latin1]{inputenc}

\title[CSCI 112: Quines and the Halting Problem]
{
CSCI 112:  Quines and the Halting Problem
}

\subtitle{} % (optional)

\author[Geoffrey Matthews]
{Geoffrey Matthews}
% - Use the \inst{?} command only if the authors have different
%   affiliation.

\begin{document}

\begin{frame}
  \titlepage
\end{frame}

%\begin{frame}
%  \frametitle{Outline}
%  \tableofcontents
%  % You might wish to add the option [pausesections]
%\end{frame}


\sect{The Halting Problem}
\begin{itemize}
\item Write a program that checks other programs for infinite loops.
\item For example, with the input:
\begin{Verbatim}[frame=single]
  while (True):
      continue
\end{Verbatim}
your program will output: \fbox{\tt Infinite loop!}
\item But with the input:
\begin{Verbatim}[frame=single]
  print(2+2)
\end{Verbatim}
your program will output: \fbox{\tt Halts!}
\pause
\item Your program must halt in every case, and so you can't just run
  the program in a simulator and check.  You might wait forever.
\pause
\item Is there another way it could be done?
\end{itemize}

\end{frame}

\sect{Paradoxical sentences}
\centerline{\bf This sentence is false.}

\vfill
\pause

\begin{itemize}
\item If it's true, then it's false.
\pause
\item If it's false, then it's true.
\pause
\item Is self-reference the problem?
\end{itemize}
\vfill

\end{frame}

\sect{Sentences can talk about other sentences}
\vfill
\begin{itemize}
\item {\bf ``I love you!''} is what I want to hear.
\item {\bf ``Cats can fly.''} is false.
\item {\bf ``my mother hates the''} is a sentence fragment.
\item  Appending {\bf``Cats cannot''} and {\bf``fly''} yields a truth.
\end{itemize}
\vfill

\end{frame}

\sect{The Quine sentence}

\centerline{ \parbox{3in}{{\bf ``yields falsehood when appended to its own quotation''}
  yields falsehood when appended to its own quotation.}}
\vfill
\pause
\begin{itemize}
\item This sentence manages to talk about itself without using
  self-reference. 
\item It only refers to the quoted sentence fragment, not itself.
\item The sentence says something about what happens when you do
  something to the sentence fragment.
\item When you do what the sentence says, you get the sentence itself.
\pause
\item Is this sentence true or false?
\pause
\begin{itemize}
\item If it's true, then it's false.
\item If it's false, then it's true.
\end{itemize}
\end{itemize}
\vfill
\end{frame}


\sect{The Quine sentence}

\centerline{ \parbox{3in}{{\bf ``yields falsehood when appended to its own quotation''}
  yields falsehood when appended to its own quotation.}}
\vfill

\begin{itemize}
\item The beauty of the Quine sentence is that it contains a {\em
  recipe}, or a {\em set of instructions}, that, when carried out,
  result in a paradoxical situation.
\pause
\item Computers can also talk about what happens when you carry out
  instructions! 
\end{itemize}

\end{frame}
\sect{Programs can analyze or even run other programs}

\begin{itemize}
\item A syntax checker for {\em Python} can be written in {\em Python}.
\begin{itemize}
\item This program could check its own syntax!
\end{itemize}
\item An interpreter for {\em Python} can be written in {\em Python}.
\begin{itemize}
\item This program could interpret itself!
\end{itemize}
\end{itemize}
\end{frame}

\sect{Programs analyzing themselves}
\begin{itemize}
\item But when we run a program on itself, we assume that, while the
  program is running, there is another copy of the program stored
  somewhere.
\item Even this is not necessary!  We can use Quine's trick!
\item Any program can be modified so that, before it does it's usual
  job, it prints a copy of its own source code.
\item Such a program is called a {\em Quine}.
\end{itemize}
\end{frame}

\sect{Python Quines}
\begin{itemize}
\item Consider first this program.  Its output is below.
\end{itemize}


\scriptsize
\begin{Verbatim}[frame=single]
#data = """Some random
text of whatever form.
Bla bla bla."""
def printDataAsData(data):
    print('data = ""' + '"' + data + '""' + '"')
def printDataAsProgram(data):
    print(data)
printDataAsData(data)
printDataAsProgram(data)
print(2+2)
\end{Verbatim}

\scriptsize
\begin{Verbatim}[frame=single]
data = """Some random
text of whatever form.
Bla bla bla."""
Some random
text of whatever form.
Bla bla bla.
4
\end{Verbatim}

\end{frame}

\sect{Scheme Quines}
\begin{itemize}
\item Clearly we could replace the text string with
anything---even the remaining text of the program.
\end{itemize}

{
\scriptsize
\begin{Verbatim}[frame=single]
data = """def printDataAsData(data):
    print('data = ""' + '"' + data + '""' + '"')
def printDataAsProgram(data):
    print(data)
printDataAsData(data)
printDataAsProgram(data)
print(2+2)"""
def printDataAsData(data):
    print('data = ""' + '"' + data + '""' + '"')
def printDataAsProgram(data):
    print(data)
printDataAsData(data)
printDataAsProgram(data)
print(2+2)
\end{Verbatim}
}
\begin{itemize}
\item Why didn't I show you the output of this program?
\end{itemize}

\end{frame}

\sect{Quines}
\begin{itemize}
\item  Clearly adding two and two could be replaced with {\em any}
  program.
\item It is thus easy to see that {\em any} program could be modified
  to print its own source code before doing anything else.
  \item This can be done in any language:  
  \url{https://www.nyx.net/~gthompso/quine.htm}
\item Thus, any program that analyzes other programs, {\em can 
  also analyze itself!}
\item We can use this trick to make some paradoxical looking programs. 
\end{itemize}
\end{frame}

\sect{The Halting Problem Again}
\begin{itemize}
\item Suppose the halting problem is solvable, and program {\em P} can
  analyze any program to determine if it halts.  Program {\em P}
  always halts.
\pause
\item Now we build a program {\em Q}, which uses {\em P} as a module.
\begin{itemize}
\pause
\item Program {\em Q} first obtains a copy of itself, $Q'$.
\pause
\item Program {\em Q} now runs {\em P} on $Q'$.
\pause
\item If {\em P} says that the copy runs forever, then halt.
\item Else loop forever.
\end{itemize}
\pause
\item  What does {\em P} say about {\em Q}?
\begin{itemize}

\item If {\em P} says {\em Q} halts, then {\em Q} runs forever.

\item If {\em P} says {\em Q} runs forever, then {\em Q} halts.
\end{itemize}
\end{itemize}
\pause

\vfill

\centerline{\bf\large The halting problem is not solvable!}

\vfill

\end{frame}

\sect{{\em All} problems about computer programs are unsolvable!}

\begin{itemize}
\item Suppose  program {\em P} can
  analyze any program to determine if it prints ``hello''.  Program {\em P}
  always halts.
  
\item Now we build a program {\em Q}, which uses {\em P} as a module.
\begin{itemize}

\item Program {\em Q} first obtains a copy of itself, $Q'$.

\item Program {\em Q} now runs {\em P} on $Q'$.

\item If {\em P} says that the copy prints ``hello'', then print ``goodbye''.
\item Else print ``hello''.
\end{itemize}

\item  What does {\em P} say about {\em Q}?
\begin{itemize}

\item If {\em P} says {\em Q} prints ``hello'', then {\em Q} prints ``goodbye''.

\item If {\em P} says {\em Q} does not print ``hello'', then {\em Q} prints ``hello''.
\end{itemize}
\end{itemize}

\vfill

{\bf\large No computer program can determine anything interesting about 
computer programs!}

\hfill {\sl\large ---Rice's Theorem, 1951}


\vfill

\end{frame}
\end{document}
\end
